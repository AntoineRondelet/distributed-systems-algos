\chapter{Conclusion}

À l'heure où des données sont produites chaque instant en proportions considérables, l'étude de Ceph comme solution de stockage distribué nous a permis de mesurer l’enjeu des systèmes distribués. Alors qu'il existe un large panel de solutions de stockage, Ceph se distingue de ses concurrents en adoptant une approche novatrice. Sa philosophie a comme fondement l'élimination de tous les points de défaillance uniques et de toute composante qui pourrait constituer un goulot d'étranglement. Aussi, l'idée est de s'approcher au maximum d'un système uniforme où le comportement est semblable d'une machine à une autre. Ceph a pour socle RADOS, composé d'un de nœuds de stockage et de nœuds moniteurs. Ce système distribué prend en charge la distribution des données au travers du cluster de machines. L'algorithme CRUSH permet de distribuer les données de façon homogène et décentralisée.

L'étude de Ceph nous a permis de prendre du recul sur les notions étudiées en SR05 ce semestre. Ainsi, nous avons pu observer la manière avec laquelle étaient traitées les problématiques de distribution de données, la gestion du temps entre les machines, ainsi que les élections et la reprise sur erreurs.

Finalement, la phase d'étude s'est achevée par le déploiement d'un cluster Ceph. L'infrastructure mise en place vise à démontrer le potentiel de Ceph, et a été pour nous l'occasion d'observer effectivement les comportements étudiés. Cette étape s'inscrit dans la continuité de l'étude théorique et conclut parfaitement ces semaines de travail.

De nos jours, des objets \og{}\emph{connectés}\fg{} sont produits et installés massivement aux quatre coins du monde. Au-delà des questions éthiques que cela soulève, il est pertinent de s'interroger sur les méthodes de stockage de telles quantités de données. Les systèmes de stockage actuels vont rencontrer des limites matérielles et devront nécessairement s'adapter pour répondre aux contraintes modernes. Bien que Ceph parvienne à se distinguer de ses concurrents, certains aspects, dont la communication client/cluster, la sécurité et le contrôle des accès doivent être améliorés pour faire de cette solution open-source une solution entièrement aboutie et, peut-être, dominer le marché du stockage distribué.