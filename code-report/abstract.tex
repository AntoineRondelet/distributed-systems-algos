\addchap*{Résumé}

Les technologies de l'information et de la communication subissent, depuis plusieurs années, une profonde transformation. Il est désormais commun de croiser des termes génériques tels que \textit{cloud}, \textit{big data} ou \textit{IaaS}\footnote{Pour Infrastucture as a Service, un dérivé du \textit{cloud computing} permettant de louer une infrastructure distante.}.

Ces termes revêtent le même besoin sous-jacent : la nécessité pour les systèmes concernés d'être \textbf{disponibles}, \textbf{performants}, \textbf{accessibles} et \textbf{évolutifs}. En particulier, ces systèmes s'appuient tous sur une composante de \textbf{stockage} qui conditionne leur bon fonctionnement.

Tandis qu'auparavant, les données étaient souvent centralisées sur un seul outil de stockage -- qu'il s'agit d'un serveur unique ou d'une ferme de serveurs --, cette solution n'est plus viable pour ces nouvelles technologies. 

En effet, d'une part, le fait qu'une donnée ne soit accessible que par un seul biais induit l'existence systématique de \textit{goulots d'étranglement}. D'autre part, le besoin de disponibilité ne tolère pas les pannes ni la perte de données.

Pour palier les inconvénients des systèmes de stockages classiques, de nombreuses solutions dites de \textbf{stockage distribué} sont apparues. Dans ce document, nous justifions de la pertinence de ces solutions et nous concentrons sur une technologie en particulier : \textbf{Ceph}.

Après avoir étudié ses fondements conceptuels et les algorithmes déployés, nous démontrons son bon fonctionnement en simulant une panne en situation réelle.