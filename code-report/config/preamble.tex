% Améliorations de classisthesis
\usepackage{arsclassica}

% Import des packages
\usepackage[a4paper,hmarginratio=1:1]{geometry}
\usepackage[T1]{fontenc}
\usepackage{geometry}
\usepackage[utf8]{inputenc}
\usepackage[french]{babel}
\usepackage{amsmath,amssymb}
\usepackage{calc}
\usepackage{listings}
\usepackage{graphicx}
\usepackage{subfig}
\usepackage{lipsum}
\usepackage{shapepar}
\usepackage{amsmath} 
\usepackage{pifont}
\usepackage[eulerchapternumbers,subfig,beramono,eulermath,pdfspacing,listings]{classicthesis}
\usepackage[onehalfspacing]{setspace}
\usepackage{pdfpages}
\usepackage{fancyvrb}
\usepackage{enumitem}
\usepackage{float}
\usepackage{tcolorbox}

% Configuration de l'agencement relatif à la langue française
\index{babel}
\index{french}
\frenchspacing
\renewcommand*{\FrenchLabelItem}{\color{halfgray}$\bullet$}

% Abréviations
\newcommand{\ie}{\textit{i.e.~}}
\newcommand{\eg}{\textit{e.g.~}}

% Espacement entre les paragraphes
\baselineskip=12pt
\setlength{\parskip}{\baselineskip}

% Réduction de l'espacement après les sections, sous-sections, et sous-sous-sections.
\titlespacing\chapter{0pt}{0pt plus 4pt minus 2pt}{25pt plus 2pt minus 2pt}
\titlespacing\section{0pt}{0pt plus 4pt minus 2pt}{0pt plus 2pt minus 2pt}
\titlespacing\subsection{0pt}{0pt plus 4pt minus 2pt}{0pt plus 2pt minus 2pt}
\titlespacing\subsubsection{0pt}{0pt plus 4pt minus 2pt}{0pt plus 2pt minus 2pt}

% Esthétique générale du code
% Nouvelles couleurs pour le code
\definecolor{comments}{RGB}{116,113,94}
\definecolor{bg}{RGB}{49,55,57}
\definecolor{kw}{RGB}{249,39,114}
\definecolor{var}{RGB}{102,217,238}
\definecolor{string}{RGB}{230,218,116}
\definecolor{digit}{RGB}{174,129,255}
\definecolor{darkblue}{rgb}{0.0,0.0,0.6}

% Configuration de listings
\usepackage{listings}
\lstset{numbers=left, numberstyle=\tiny, breaklines=true, basicstyle=\small, basicstyle=\ttfamily,columns=fullflexible, showstringspaces=false, captionpos=b, frame=TrBl, } 

% Base des règles du code
\lstset{upquote=true,keywordstyle=\color{var}\ttfamily,stringstyle=\color{string}\ttfamily,backgroundcolor=\color{bg},rulecolor=\color{bg}, commentstyle=\color{comments}\ttfamily,emph={int,char,double,float,unsigned,return,true,false},emphstyle={\color{kw}},basicstyle=\ttfamily\footnotesize\color{white},tabsize=3,literate=
	*{0}{{{\color{digit}0}}}1
	{1}{{{\color{digit}1}}}1
	{2}{{{\color{digit}2}}}1
	{3}{{{\color{digit}3}}}1
	{4}{{{\color{digit}4}}}1
	{5}{{{\color{digit}5}}}1
	{6}{{{\color{digit}6}}}1
	{7}{{{\color{digit}7}}}1
	{8}{{{\color{digit}8}}}1
	{9}{{{\color{digit}9}}}1
	{.0}{{{\color{digit}.0}}}1
	{.1}{{{\color{digit}.1}}}1
	{.2}{{{\color{digit}.2}}}1
	{.3}{{{\color{digit}.3}}}1
	{.4}{{{\color{digit}.4}}}1
	{.5}{{{\color{digit}.5}}}1
	{.6}{{{\color{digit}.6}}}1
	{.7}{{{\color{digit}.7}}}1
	{.8}{{{\color{digit}.8}}}1
	{.9}{{{\color{digit}.9}}}1
}

% Box de couleur
\definecolor{PimpedBrown}{RGB}{245,222,179}
\definecolor{PimpedRed}{RGB}{222,102,102}
\newtcolorbox{PimpedBox}{colback=PimpedBrown,colframe=brown}
\newtcolorbox{WarningBox}{colback=PimpedRed,colframe=brown}

% JavaScript
\lstdefinelanguage{JavaScript}{
  keywords={typeof, new, true, false, catch, function, return, null, catch, switch, var, if, in, while, do, else, case, break},
  comment=[l]{//},
  morecomment=[s]{/*}{*/},
  morestring=[b]',
  morestring=[b]"
}

% Configuration du symbole Warning
%Source :https://tex.stackexchange.com/questions/159669/how-to-print-a-warning-sign-triangle-with-exclamation-point
\newcommand\Warning{%
 \makebox[1.4em][c]{%
 \makebox[0pt][c]{\raisebox{.1em}{\small!}}%
 \makebox[0pt][c]{\color{red}\Large$\bigtriangleup$}}}%